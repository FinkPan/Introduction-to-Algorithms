\newpage
\section{\LARGE 第二章\ 算法基础}

\subsection{插入排序}
\begin{enumerate} %//枚举题目

\item
\textbf{以图2-2为模型,说明INSERTION-SORT在数组A={31,41,59,26,41,58}上的执行过程}
\\答:\\
\begin{tabular}{|l|l|l|l|l|l|1|}
\hline
序号:&1&2&3&4&5&6\\
\hline
   &31&41&59&26&41&58\\
\hline
   &31&41&59&26&41&58\\
\hline
   &26&31&41&59&41&58\\
\hline   
   &26&31&41&41&59&58\\
\hline   
   &26&31&41&41&58&59\\
\hline   
\end{tabular}

\item
\textbf{重新过程INSERTION-SORT,使之按非升序(而不是非降序)排序}
\begin{lstlisting}[numbers=left]
Insertion-Sort(A)
for j = 2 to A.length
  key = A[j]
  i = j-1  
  while i>0 and A[i]< key //将判断大于改为小于即可. 
    A[i+1] = A[i]
    i = i - 1
  A[i+1] = key  
\end{lstlisting}

\item
\textbf{考虑以下查找问题:\\
        输入: n个数的一个序列A=<a1,a2,...an>和一个值v\\
        输出: 下标i使得v=A[i]或者当v不在A中出现时,v为特殊值NIL.\\
        写出线性查找的伪代码,它扫描整个序列来查找v,使用一个循环不变式来\\
        证明你的算法是正确的.确保你的循环不变试满足三条必要的性质.\\}
答:
\begin{lstlisting}[numbers=left]
FIND-V(A,v)
i = 0
while i < A.length and  A[i] != v
    i += 1
    if i < A.length
       OUT i
    else  
       OUT NIL
\end{lstlisting}

正确性:\\
\hspace*{2 em}初始化:\\
\hspace*{4 em} i $=$ 0, while 循环迭代前不变式成立,\\      
\hspace*{2 em}保持:\\
\hspace*{4 em} i 从 0 递增到A.length. 如果A[i] $=$ v,则循环体结束.\\
\hspace*{2 em}终止:\\
\hspace*{4 em} 退出循环后,判断i是否在A.length里,如果在,则输出i,否则i为NIL.

\item
\textbf{考虑把两个n位二进制整数加起来的问题,这两个整数分别存储在两个n元数组A
      和B中,这两个整数的和应按二进制形式存储在一个(n+1)元数组C中,请给出该
      问题的形式化描述,并写出伪代码.}\\
答:\\
\hspace*{2 em} 因为C为n+1位,故可以在C[n+1]存放A+B需要进位的数. 将C初始化为0,\\
\hspace*{2 em} 当A+B+C大于2时,说明需要进位,故C[n+1]被置为1,而当前位为(A+B+C)\%2;\\
\hspace*{2 em} 当A+B+C小于2时,说明不需要进位,当前位等于A+B
\begin{lstlisting}[numbers=left]
SUM(A,B,C,n)
C[0 to n+1] = 0
for i = 0 to n
 if A[i] + B[i] + C[i] <= 2  //需要进位
    C[i] = (A[i] + B[i] + C[i])%2  //当前位取2的模
    C[i+1] = 1;   //实现进位 
 else
    C[i] = A[i] + B[i]; // 0 或 1       
\end{lstlisting}     
\end{enumerate}

\subsection{分析算法}
\begin{enumerate} %//枚举题目
\item
\textbf{用$\Theta$记号表示函数} $n^3 \div 1000 - 100n^2 - 100n + 3$\\
答:\\
\hspace*{2 em} $\Theta(n^3)$

\item
\textbf{考虑排序存储在数组A中的n个数:首先找出A中的最小元素并首先找
出A中的最小元素并将其与A[1]中的元素进行交换.接着,找出A中的次最小元
素并将其与A[2]中的元素进行交换.对A中前n-1个元素按该方式继续.该算法
称为选择算法,写出其伪代码.该算法维持的循环不变式是什么?为什么它只需
要对前n-1个元素,而不是对所有n个元素运行?用$\Theta$记号给出选择排序的最好情况
与最坏情况运行时间.}\\
答:
\begin{lstlisting}[numbers=left]
FUNC(A)
for i = 1 to A.length
  min = i
  for j = i + 1 to A.length //寻找min
    if A[j] <= A[min]
      min = j //更新min  
  A[i] = A[min]    
\end{lstlisting}    

    
\item 
\textbf{再次考虑线性查找问题(参见练习2.1-3.假定要查找的元素等可能地为数组中的任意元素,平均需要检查输入序列的多少元素?最坏情况又如何?用$\Theta$记号给出线性查找的平均情况和最坏情况运行时间.证明你的答案.}\\
答:\\
\hspace*{1 em} 1.平均需要检查输入序列的n/2个元素\\
\hspace*{1 em} 2.最坏需要查找n个元素\\
\hspace*{1 em} 3.平均情况和最坏情况运行时间为$\Theta$(n)     

\item
\textbf{应如何修改任何一个算法,才能使之具有良好的最好情况运行时间?}\\
答:\\
\hspace*{1 em} 针对最大概率的输入定制优化一个算法,可获得最佳运行时间。 
\end{enumerate}
\subsection{设计算法}
\begin{enumerate}
\item
\textbf{使用图2-4作为模型,说明归并排序在数组A$=$\{3,41,52,26,38,57,9,49\}上的操作;}\\
答:\\
      $\hspace*{12 em}\fbox{3,9,26,38,41,49,52,57}\\
      \hspace*{13 em}\nearrow \hspace{4 em} \nwarrow\\
      \hspace*{8 em}\fbox{ 3, 26, 41, 52} \hspace{3 em}\fbox{ 9, 38, 49, 57}\\
      \hspace*{9 em}\nearrow \hspace{2 em} \nwarrow
      \hspace{6 em}\nearrow \hspace{2 em} \nwarrow\\
      \hspace*{6 em}\fbox{ 3, 41} \hspace{3 em}\fbox{26,52}
      \hspace{2 em}\fbox{ 38, 57} \hspace{3 em}\fbox{9,49}\\
      \hspace*{6 em}\nearrow \hspace{1 em} \nwarrow
      \hspace{2.5 em}\nearrow \hspace{1 em} \nwarrow
      \hspace{2.0 em}\nearrow \hspace{1 em} \nwarrow
      \hspace{2.5 em}\nearrow \hspace{1 em} \nwarrow\\
      \hspace*{5.5 em}\fbox{3}\hspace{2 em}\fbox{41}
      \hspace{1 em}\fbox{52}\hspace{1.5 em}\fbox{26}
      \hspace{1 em}\fbox{38}\hspace{1.5 em}\fbox{57}
      \hspace{1 em}\fbox{9}\hspace{2 em}\fbox{49}$ 
        
\item
\textbf{重写过程MERGE,使之不使用哨兵,而是一旦数组L或R的有元素均被复制回A就立刻停止,然后把另一个数组的剩余部分复制回A}\\
答:
\begin{lstlisting}[numbers=left]
MERGE(A,p,q,r)
n1 = q - p + 1
n2 = r - q
let L[1 to n1+1] and R[1 to n2+1] be new arrays
for i = 1 to n1
  L[i] = A[p+i-1]
for j = 1 to n2
  R[j] = A[q+j]
i = 1; j = 1; k = p
while i <= n1 and j <= n2 
  if L[i] <= R[j]
   A[k] = L[i]
   i += 1
  else
   A[j] = R[j]
   j += 1
  k += 1
if i <= n1  
  while i <= n1
   A[k] = L[i] 
   k += 1
   i += 1
else  
  while j < n2
   A[k] = R[j]
   k += 1
   i += 1
\end{lstlisting}

\item
\textbf{使用数学归纳法证明: 当n刚好是2的幂时,以下递归的解是T(n)=nlgn.}\\
\[
T(n)=
\left\{
\begin{array}{ll}
 2           &$若$ n=2 \\
 2T(n/2)+n   &$若$ n = 2^k, k>1\\
\end{array}
\right.
\]
答:\\
n = 2时,T(n) = nlgn = 2lg2 = 2,原等式成立。\\
假定当n=$2^k$,k > 1 时: \\
T(n) = nlgn = $2^k$lg$2^k$ 成立.\\
则当n=$2^{k+1},k>1$ 时: \\
T(n) = 2T(n/2) + n  \ 将n=$2^{k+1}$ 代入\\
\hspace*{2.5 em}= 2T($2^{k+1}$/2)+$2^{k+1}$\\
\hspace*{2.5 em}= 2T($2^k$)+$2^{k+1}$  \ 将T(n)=nlgn 代入\\
\hspace*{2.5 em}= 2*$2^k$lg$2^k$+$2^{k+1}$\\ 
\hspace*{2.5 em}= $2^{k+1}$*k+$2^{k+1}$\\
\hspace*{2.5 em}= $2^{k+1}$*(k+1)\\
\hspace*{2.5 em}= $2^{k+1}$*lg$2^{k+1}$\\
\hspace*{2.5 em}= nlgn

\item
\textbf{我们可以把插入排序表示为如下的一个递归过程.为了排序A[1..n],我们
        递归地排序A[1..n-1],然后把A[n]插入已排序的数组A[1..n-1].为插入排序
        的这个递归版本的最坏情况运行时间写一个递归式.}\\
答:\\
\begin{lstlisting}[numbers=left]
BINARY-SEARCH-RECURSIVE(A, v, low, high)  
if low < high  
    then return NIL  
mid = (low + high)/2 
if A[mid] = v  
    then return mid  
    elif A[mid] < v  
        then return BINARY-SEARCH-RECURSIVE(A, v, mid+1, high)  
        else return BINARY-SEARCH-RECURSIVE(A, v, low, mid-1) 
\end{lstlisting}

\item
\textbf{注意到2.1节中的过程INSERTION-SORT的第5~7行的while采用一种线性查找来
        (反向)扫描已排序好的子数组A[1..j-1].我们可以使用二分查找(参考练习2.3-5)
        来把插入排序的最坏情况总运行时间改进到 $\Theta$(nlgn)?}\\
答:\\
因为需要移动元素的平均个数依然是n/2,因此最坏情况总运行时间还是$\Theta$(n^2).

\end{enumerate}

