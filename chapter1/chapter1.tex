\section{第一章\ 算法在计算中的作用}

\subsection{算法}
\begin{enumerate} %//枚举题目
\item
\textbf{给出现实生活中需要排序的一个例子或者现实生活中需要计算凸壳的一个例子。}
\\答:\\ 排序: 图书馆查找图书需要根据用户需求按书名. 凸壳: 计算N个像控点最大面积应该是凸壳.
\item
\textbf{除速度外,在真实环境中还可能使用那些其他有关效率的量度.}
\\答:\\ 程序效率:计算复杂度; 工程效率:工程进度
\item
\textbf{选择一种你以前已知的数据结构,并讨论其优势和局限}
\\答:\\ 链表: 插入,删除方便; 但查找不方便.需要一个个遍历.
\item
\textbf{前面给出的最短路径与旅行商问题有哪些相似之处?又有哪些不同.}
\\答:\\
1:两者都是寻找最短路径。\\ 2.最短路径只需要找到最短的联通即可,而旅行商需要每个节点都要遍历一次.
\item
\textbf{提供一个现实生活的问题,其中只有最佳解才行,然后提供一个问题,其中近似最佳的一个解也足够好.}
\\答:\\
估算H单应,或者拟合各种线.
\end{enumerate}

\subsection{作为一种技术的算法}
\begin{enumerate} %//枚举题目
\item
\textbf{给出应用层需要算法内容的应用的一个例子,并讨论涉及的算法的功能.}
\\答:
\\Hash表需要根据给定KEY生成对应地址方便程序快速查找并定位KEY对应的内容.
\item
\textbf{假设我们正比较插入排序与归并排序在相同机器上的实现,对规模为$n$的输入,插入排序运行$8n^2$步,而归并排序运行$64nlgn$步,问对那些$n$值,插入排序优于归并排序?}
\\答:
\\问题等价解\ $8n^2 < 64nlgn  \ n$是整数,用$n=\{1\to n\}$一个个代入解得:
\\ \hspace*{10 mm}$n=1\ :\hspace{10 mm}  8>0$
\\ \hspace*{10 mm}$n=2\ :\hspace{10 mm} 32<128$
\\ \hspace*{15 mm} ....
\\ \hspace*{10 mm}$n=43:\hspace{10 mm} 14762<14933$
\\ \hspace*{10 mm}$n=44:\hspace{10 mm} 15488>15373$
\\即 $2\leq n \leq 43$ 的时候插入排序优于归并排序.
\item
\textbf{$n$的最小值为何值时,运行时间为$100n^2$的一个算法在相同机器上快于运行时间为$2^n$的另一个算法?}
\\答:
\\问题等价解 $100n^2 < 2^n \ n$是整数, 也是简单代入解得:当$n \geq 15$时满足

\end{enumerate}
